%----------------------------------------------------------------------------------------
%	ELECTRON LEVELS IN A PERIODIC POTENTIAL
%----------------------------------------------------------------------------------------
\chapter{Electron Levels in a Periodic Potential}

\listoftodos

\section{To Do:}
	I think we should organize it by learning objectives, introduction, main text, and summary, as the prof has outlined in the template.

	I think we should go through the notes from his slides to see if there is anything valuable there.

	We need to check the notation for $K$ and $G$ since we should be consistent with the notes and not A\&M

	we might want to write a summary of the main points and equations

	I made the $\mathbf{U_K}$ and the $\mathbf{c_q}$ as vectors, but I don't know if they should be.

	Spell check/ syntax

	Include a section about why the reciprocal space becomes so important here.

\section{Preface}
	\todo{We should somehow rename this section}

	The writers' goals are:
	\begin{enumerate}
		\item To give a clearer picture of the direction of the proof before finding the result
		\item To explain things that were not fully explained, and answer questions we ourselves had when reading the document
		\item To add extra insight and applications to A\&M
	\end{enumerate}

	\todo{I think that the more sources we have, the better it looks.}

	\begin{quotation}
		``When I started to think about it, I felt that the main problem was to explain how the electrons could sneak by all the ions in a metal so as to avoid a mean free path of the order of atomic distances. Such a distance was much too short to explain the observed resistances, which even demanded that the mean free path become longer and longer with decreasing temperature. But Heitler and London had already shown how electrons could jump between two atoms in a molecule to form a covalent bond, and the main difference between a molecule and a crystal was only that there were many more atoms in a periodic arrangement. To make my life easy, I began by considering wave functions in a one-dimensional periodic potential. By straight Fourier analysis I found to my delight the the wave differed from a a plane wave of free electron only by a periodic modulation.
		This was so simple that I didn’t think it could be much of a discovery, but when I showed it to Heisenberg he said right away, “That’s it.” Well, that wasn’t quite it yet, and my calculations were only completed in the summer when I wrote my thesis on 'The Quantum Mechanics of Electrons in Crystal Lattices.''
	\end{quotation}
	- Felix Bloch


	These waves are now referred to as "Bloch Waves" or "Bloch States", and they are a fundamental building bloch of condensed matter physics. This was already known in mathematics as Floquet's Theorem, but Bloch discovered its immense implications for quantum mechanics.
	Biographical Memoirs, Volume 64 \todo{cite properly}

\section{Introduction}
	Recall the free electron and independent electron approximations from the assumptions of the Sommerfeld Model:

	\begin{enumerate}
		\item \textbf{Free electron approximation.} The positive ions are there only to maintain charge neutrality. The interaction between ions and valence electrons is neglected except in boundary conditions.
		\item \textbf{Independent electron approximation.} The interactions between electrons are neglected.
	\end{enumerate}


	Although the free electron approximation provides insight into properties of metals, such as heat capacity, thermal conductivity, and electrical conductivity, it does not account for the differences between metals, semiconductors, and insulators, and it cannot account for positive values of the Hall coefficient. In addition, conduction electrons in the metal do not correctly correspond to the number of valence electrons in free atoms. The purpose of this chapter is to set the stage for understanding band structure.

\section{Perfect crystal approximation}
    In this chapter, a perfect crystal is considered. But it is important to know that it is impossible to have the perfect periodicity. This means that there are always some defects existed in a crystalline material. There are different types of defects in a crystalline material such as grain boundaries, twin boundaries, dislocations, and vacancies, to name but a few. By controlling the fabrication process one could produce a single crystal material without any dislocation. However, it is impossible to produce a material without vacancies. This could be proved by evaluating the Gibbs free energy of system. Consider $X_{v}^{e}$ is equilibrium vacancy concentration which is according to the minimum Gibbs free energy of system ($\Delta G$). In order to calculate $\Delta G$ consider two components existed in the system, atoms and vacancies. From thermodynamics we have:
    \begin{equation} \label{eq:Gibbs_1}
        G_{2} = G_{1} +\Delta G
    \end{equation}
    $G_{2}$ is the final state and $G_{1}$ is the initial state.
    \begin{equation} \label{eq:Gibbs_2}
        \Delta G = \Delta H - T\Delta S
    \end{equation}
    The increase in the enthalpy of the system is directly proportional to the number of vacancies added, as the equilibrium concentration of vacancies is small and vacancy-vacancy interactions can be ignored.
    \begin{equation*}
        \Delta H = X_{v}\Delta H_{v}
    \end{equation*}
    $X_{v}$ is the mole fraction vacancies, and $\Delta H_{v}$ is the increase in enthalpy per mole of vacancies added to system.\newline
    In the case of $\Delta S$ there are to terms need to be considered. The first term is the thermal entropy per mole of vacancies, $\Delta S_{v}$. The second one which has the largest contribution, is the configurational entropy, $\Delta S_{mix}$. The total change in entropy is:
    \begin{equation*}
        \Delta S = X_{v}\Delta S_{v} + \Delta S_{mix}
    \end{equation*}
    \begin{equation*}
        \Delta S = X_{v}\Delta S_{v} + Kln(\omega)
    \end{equation*}
    \begin{equation*}
        \Delta S = X_{v}\Delta S_{v} - R(X_{v}ln(X_{v})+(1-X_{v})ln(1-X_{v}))
    \end{equation*}
    Finally, according equation \ref{eq:Gibbs_1}:
    \begin{equation} \label{eq:Gibbs_3}
        G_{2} = G_{1} + \Delta G = G_{1} + X_{v}\Delta H_{v} - TX_{v}\Delta S_{v} - RT(X_{v}ln(X_{v})+(1-X_{v})ln(1-X_{v}))
    \end{equation}
    Now the minimum of $G_{2}$ can be calculated by the differentiating equation \ref{eq:Gibbs_3}:
    \begin{equation*}
        \left.\frac{dG}{dX_{v}}\right|_{X_{v}=X_{v}^{e}} = 0
    \end{equation*}
    As mentioned above, $X_{v}^{e}$ is small, $X_{v}^{e} \ll 1$, then equilibrium concentration of vacancies is:
    \begin{equation*}
        \Delta H_{v} - T\Delta S_{v} + RTln(X_{v}^{e}) = 0
    \end{equation*}
    \begin{equation} \label{eq:vacancy}
        X_{v}^{e} = exp\frac{\Delta S_{v}}{R}.exp(\frac{-\Delta H_{v}}{RT})
    \end{equation}
    It is good to know that, practically, $\Delta H_{v}$ is in the order of 1 ev/atom.\newline
    According to equation \ref{eq:vacancy}, it is impossible to find a perfect crystal without any vacancies or defects. In the following, we consider the ideal perfect crystal with the periodic potential.

\section{Periodic Potential}

	\todo{Find something similar to figure 8.1 and include it. See latex.tex for how to include the figure}

	In a perfect crystal, the electron potential looks the same at each point on the crystal lattice. This can be expressed mathematically as:
	\begin{equation} \label{eq:U_period}
		U(\mathbf{r} + \mathbf{R}) = U(\mathbf{r})
	\end{equation}
	where $\mathbf{R}$ is the Bravais lattice vector. The scale of periodicity of the potential, $U$ ($\sim$\num{d-8} cm), is the size of a typical de Broglie wavelength of an electron in the Sommerfeld free electron model, so quantum mechanics can describe the wavefunction with this same periodicity. 

	The electronic properties of a crystal can be examined with the Schrödinger equation while maintaining the \textit{independent electron approximation}:
	\begin{equation} \label{eq:schrod}
		H\psi=\left(-\frac{\hbar^2}{2m}\boldsymbol{\nabla}^2 + U(\mathbf{r})\right)\psi = \varepsilon\psi
	\end{equation}

	These electrons are known as \textit{Bloch electrons}, and they reduce to ``free electrons'' when the periodic potential, $U(\mathbf{r})$ is equal to zero.

	Bloch proved that the solutions to the Schrödinger equation for a periodic potential must be of the form:

	\begin{equation} \label{eq:bloch_theorem}
		\psi_{n\mathbf{k}}(\mathbf{r}) 
		= e^{i\mathbf{k} \cdot \mathbf{r}}
		u_{n\mathbf{k}}(r)
	\end{equation}

	where $u_{n\mathbf{k}}(\mathbf{r})$ has the period of the crystal lattice with $u_{n\mathbf{k}}(\mathbf{r}) = u_{n\mathbf{k}}(\mathbf{r} + \mathbf{T})$
	where $n$ is a discrete index, called the band index, which is present because there are many different Bloch waves with the same $k$ (each has a different periodic component u).
	Here, $\mathbf{T}$ is a translation vector of the lattice. This result (eq. \ref{eq:bloch_theorem}), expresses the Bloch theorem:
	\begin{quotation}
		The eigenstates of the wave equation can be chosen so that, for every $\mathbf{R}$ in the Bravais Lattice, each $\psi$ has an associated wave vector, $\mathbf{k}$, such that:
	\end{quotation}

	\begin{equation} \label{eq:bloch_theorem_restate}
		\psi(\mathbf{r} + \mathbf{R})
		= e^{i\mathbf{k} \cdot \mathbf{R}}
		\psi(\mathbf{r})
	\end{equation}


	Ashcroft and Mermin provide two proofs of Bloch's Theorem, which we will reproduce below.

\section{Quantum Mechanical proof of Bloch's Theorem}
	This proof is valid even when $\psi_{\mathbf{k}}$ is degenerate, \todo{put this somewhere else. Also check about the other proof} i.e. when there are other wavefunctions with the same energy and wavevector as $\psi_{\mathbf{k}}$.

	For each Bravais lattice vector $\mathbf{R}$, let $T_{\mathbf{R}}$ be the translation operator, which, when operating on any function $f(\mathbf{r})$, shifts the argument by $\mathbf{R}$:

	\begin{equation} \label{eq:translation}
		T_{\mathbf{R}} f(\mathbf{r})
		= f(\mathbf{r} + \mathbf{R})
	\end{equation}

	Due to the periodicicity of the lattice, we can show that the translation operator commutes with the Hamiltonian:

	\begin{equation} \label{eq:trans_commute1}
		T_{\mathbf{R}} H \psi
		= H \psi(\mathbf{r} + \mathbf{R})
		= H(\mathbf{r}) \psi(\mathbf{r} + \mathbf{R})
		= HT_{R} \psi
	\end{equation}

	Note that the application of two successive translations does not depend on the order in which they are applied, since for any $\psi(\mathbf{r})$

	\begin{equation} \label{eq:trans_commute2}
		T_{\mathbf{R}} T_{\mathbf{R'}} \psi(\mathbf{r})
		= T_{\mathbf{R'}} T_{\mathbf{R}} \psi(\mathbf{r})
		= \psi(\mathbf{r} + \mathbf{R} + \mathbf{R'})
	\end{equation}

	Therefore
	\begin{equation} \label{eq:trans_addition}
		T_{\mathbf{R}} T_{\mathbf{R'}}
		= T_{\mathbf{R'}} T_{\mathbf{R}}
		= T_{\mathbf{R + R'}}
	\end{equation}

	Because eq. (\ref{eq:trans_commute1}) holds identically for any function $\psi$, $T_{\mathbf{R}}$ and $H$ commute:

	\begin{equation} \label{eq:trans_commute_relation}
		T_{\mathbf{R}} H = H T_{\mathbf{R}}
	\end{equation}

	A theorem of quantum mechanics \todo{maybe we should say which theorem. see AM subnote on page 134.} allows us to choose eigenstates of $H$ that are simultaneous eigenstates of all the $T_{\mathbf{R}}$:

	\begin{eqnarray} 
		H\psi &=& \varepsilon\psi \nonumber \\
		T_{R} \psi &=& c(\mathbf{R}) \psi \label{eq:eigenstates}
	\end{eqnarray}

	The eigenvalues $c(\mathbf{R})$ of the translation operators are related in the following way:
	\begin{equation} \label{eq:c_relation}
		c(\mathbf{R} + \mathbf{R'}) 
		= c(\mathbf{R}) c(\mathbf{R}')
	\end{equation}

	This can be seen from eq. (\ref{eq:trans_addition}):
	\begin{equation} \label{eq:derive_crelation1}
		T_{\mathbf{R}'} T_{\mathbf{R}} \psi
		= c(\mathbf{R}) T_{\mathbf{R}} \psi
		= c(\mathbf{R}) c(\mathbf{R}')
	\end{equation}

	Eq. (\ref{eq:trans_addition}) can \emph{also} be used to formulate the eigenvalues as such:
	\begin{equation}
		T_{\mathbf{R}'} T_{\mathbf{R}} \psi
		= T_{\mathbf{R} + \mathbf{R'}} \psi
		= c(\mathbf{R} + \mathbf{R'}) 
	\end{equation}

	In order to satisfy Bloch's theorem, in the form of eq. (\ref{eq:bloch_theorem_restate}), we can come up with an expression for the exponential, $e^{i\mathbf{k} \mathbf{R}}$ in terms of the eigenvalue, $c(\mathbf{R})$ through the following derivation.

	We first write an expression for $\mathbf{R}$ as the Bravais lattice vector given by
	\begin{equation} \label{eq:r_express}
		\mathbf{R} = n_{1} \mathbf{a}_{1}
		+ n_{2} \mathbf{a}_{2}
		+ n_{3} \mathbf{a}_{3}
	\end{equation}

	where $\mathbf{a_{i}}$ are the three primitive vectors of the Bravais lattice, which can be captured by $\mathbf{c}$: \todo{Can someone please explain how we can express $c$ this way because I do not get it! Thanks.}

	\begin{equation} \label{eq:c_exp}
			c(\mathbf{a}_{i}) = e^{2 \pi i x_{i}}
	\end{equation}

	by a suitable choice \todo{yeah I don't know what ``suitable'' means. Please explain} of $x_i$. From successive applications of \ref{eq:c_relation} that

	\begin{equation} \label{eq:c_mult}
		c(\mathbf{R})
		= c(\mathbf{a}_1)^{n_1}
		c(\mathbf{a}_2)^{n_2}
		c(\mathbf{a}_3)^{n_3}
	\end{equation}

	and since
	\begin{equation} \label{eq:k_with_bi}
		\mathbf{k} = x_1 \mathbf{b}_1
		+ x_2\mathbf{b}_2
		+ x_3 \mathbf{b}_3
	\end{equation}

	where $b_i$ are the reciprocal lattice vectors satisfying
	\begin{equation} \label{eq:recip_b}
		\mathbf{b_i \cdot a_j}
		= 2 \pi \delta_{ij}
	\end{equation}

	this means that
	\begin{equation} \label{eq:c(R)}
		c(\mathbf{R}) 
		= e^{i \mathbf{k} \cdot \mathbf{R}}
	\end{equation}

	We can thus choose the eigenstates $\psi$ of $H$ so that for every Bravais lattice vector $\mathbf{R}$, we can obtain Bloch's theorem:
	\begin{equation} \label{eq:bloch_proof_qm}
		T_{\mathbf{R}} \psi
		= \psi(\mathbf{r} + \mathbf{R}) \psi
		= e^{i \mathbf{k} \cdot \mathbf{R}}
	\psi(\mathbf{r})
	\end{equation}
	\todo{I think we can add some things from the PDF I sent. We should find out what sources the PDF used and include that}

\section{Form of the Bloch Wave Vector}
	\todo{Note here that I renamed this section from the BVK Boundary Conditions.}
	To demonstrate that the wave vector $\mathbf{k}$ must be real, we must impose the appropriate boundary conditions. The Born-Von Karman boundary condition was originally presented earlier as
	\begin{eqnarray} \label{eq:bvk_ch2}
		\psi(x,y,z+L) &=& \psi(x,y,z), \\ \nonumber
		\psi(x,y+L,z) &=& \psi(x,y,z), \\
		\psi(x+L,y,z) &=& \psi(x,y,z) \nonumber
	\end{eqnarray}

	but these boundary conditions apply to a cubic Bravais lattice with side L that is an integral multiple of the lattice constant, $a$. A general form of eq. (\ref{eq:bvk_ch2}) is 
	\begin{alignat}{2}\label{eq:bvk}
		\psi(\mathbf{r} + N_i \mathbf{a}_i)
		&= \psi(\mathbf{r}),
		&\quad
		i &= 1,2,3
	\end{alignat}

	The $\mathbf{a}_i$ are the primitive vectors, and the $N_i$ are all integers of order $N^{1/3}$, where $N=N_1N_2N_3$ is the total number of primitive cells in the crystal.

	Applying Bloch's theorem, eq. (\ref{eq:bloch_theorem_restate}) to the boundary condition, eq. (\ref{eq:bvk}), \todo{can someone please explain where the n comes from?} we obtain

	\begin{alignat}{2}\label{eq:bloch_applied}
		\psi_{n\mathbf{k}}
		(\mathbf{r} + N_i \mathbf{a}_i)
		&= 
		e^{i N_i \mathbf{k} \cdot \mathbf{a}_i}
		\psi(\mathbf{r}),
		&\quad
		i &= 1,2,3
	\end{alignat}

	which requires that

	\begin{equation} \label{eq:requirement}
		e^{2 \pi i N_i x_i} = 1
	\end{equation}

	meaning
	\begin{alignat}{2} \label{eq:xi}
		x_i &= \frac{m_i}{N_i},
		& \quad
		m_i\text{ integral}
	\end{alignat}

	\todo{I think I need to edit this section a bit more because it is just copied- there are no additional explanations}

	Therefore the general form for allowed Bloch wave vectors is

	\begin{alignat}{3} \label{eq:wave_vector}
		\mathbf{k} =
		\sum_{i=1}^{3} \frac{m_i}{N_i}
		\mathbf{b_i}
		,
		&\quad 
		&m_i\text{ integral}
	\end{alignat}

	It follows from eq. (\ref{eq:wave_vector}) that the volume $\Delta\mathbf{k}$ of $k$-space per allowed value of $\mathbf{k}$ is just the volume of the little parallelepiped with edges $\mathbf{b}_i/N_i$:

	\begin{equation} \label{eq:deltak}
		\Delta{k}
		= 
		\frac{\mathbf{b_1}}{N_1}
		\cdot
		\left(
		\frac{\mathbf{b_2}}{N_2}
		\times
		\frac{\mathbf{b_3}}{N_3}
		\right)
		=
		\frac{1}{N}
		\mathbf{b_1}
		\cdot
		\left(
		\mathbf{b_2}
		\times
		\mathbf{b_3}
		\right)
	\end{equation}

	Since $\mathbf{b_1} \cdot (\mathbf{b_2} \times \mathbf{b_3})$ is the volume of a reciprocal lattice primitive cell, eq. (\ref{eq:deltak}) asserts that \emph{the number of allowed wave vectors in a primitive cell of the reciprocal lattice is equal to the number of sites in the crystal}.

	The volume of a reciprocal lattice primitive cell is $(2π)^3/v$. This can be proved by considering reciprocal and direct bravais lattice vectors.
	Consider $a_{1}$, $a_{2}$, and $a_{3}$  are three primitive vectors of direct bravais lattice. then the volume of direct primitive bravais lattice is:
	\begin{equation*}
		\nu = a_{1}.(a_{2}\times a_{3})
	\end{equation*}
	The reciprocal lattice vectors are:
	\begin{equation*}
		b_{1} = 2\pi.\frac{a_{2}\times a_{3}}{\nu}
	\end{equation*}
	\begin{equation*}
		b_{2} = 2\pi.\frac{a_{3}\times a_{1}}{\nu}
	\end{equation*}
	\begin{equation*}
		b_{3} = 2\pi.\frac{a_{1}\times a_{2}}{\nu}
	\end{equation*}
	The volume of reciprocal lattice is:
	\begin{equation*}
		\nu_{reciprocal} = b_{1}.(b_{2}\times b_{3})
	\end{equation*}
	\begin{equation*}
		\nu_{reciprocal} = b_{1}.(\frac{4\pi^2}{\nu^2})(a_{3}\times a_{1})\times(a_{1}\times a_{2})
	\end{equation*}
	\begin{equation}\label{eq:vol_1}
		\nu_{reciprocal} = (\frac{8\pi^3}{\nu^3})(a_{2}\times a_{3}).[(a_{3}\times a_{1})\times(a_{1}\times a_{2})]
	\end{equation}
	From cross products, we know that:
	\begin{equation}\label{eq:cross_product}
		a\times(b\times c) = (a.c)b - (a.b)c
	\end{equation}
	Now if equation \ref{eq:cross_product} is applied to the right side of equation \ref{eq:vol_1}:
	\begin{equation*}
		\nu_{reciprocal} = (\frac{8\pi^3}{\nu^3})(a_{2}\times a_{3}).[[(a_{3}\times a_{1}).a_{2}]a_{1}-[(a_{3}\times a_{1}).a_{1}]a{2}] 
	\end{equation*}
	\begin{equation*}
		\nu_{reciprocal} = (\frac{8\pi^3}{\nu^3})(a_{2}\times a_{3}).[[\nu]a_{1}-[0]a{2}]=  (\frac{8\pi^3}{\nu^3})(a_{2}\times a_{3}).(\nu a_{1})
	\end{equation*}
	\begin{equation*}
		\nu_{reciprocal} =  (\frac{8\pi^3}{\nu^2})(a_{2}\times a_{3}).(a_{1}) = \frac{8\pi^3}{\nu} 
	\end{equation*}
    where $v=V/N$ is the volume of a direct lattice primitive cell. and V is the volume of solid.
	so eq. (\ref{eq:deltak}) can be written in the alternative form:
	\begin{equation} \label{eq:freedeltak}
		\Delta \mathbf{k}
		=
		\frac{(2 \pi)^3}{V}
	\end{equation}

	which is the same as the free electron volume.
	\todo{expand upon this}

\section{Proof 2 of Bloch's Theorem}
	\todo{this section is also pretty much verbatim from the text}
	The first proof of Bloch's Theorem allowed us to formulate the expression for the Bloch wave vector as a function of the reciprocal lattice vectors. \todo{correct?} This second proof is useful in later discussions pertaining to band energies.

	Using eq. (\ref{eq:wave_vector}), one can always expand any function obeying the Born-von Karman boundary condition, (\ref{eq:bvk}) in the set of all plane waves that satisfy the boundary condition, resulting in wave vectors with the form:

	\begin{equation} \label{eq:waves_q}
		\psi(\mathbf{r})
		=
		\sum_{q}
		\mathbf{c_q}
		e^{i\mathbf{q} \cdot \mathbf{r}}
	\end{equation}

	The $\mathbf{q}$ refers to the primitive vectors in real space. \todo{PLEASE CHECK THIS STATEMENT because I don't know if this is correct!} Because the potential $U(\mathbf{r})$ is periodic in the lattice, its \emph{plane wave expansion} will only contain plane waves with the periodicity of the lattice, thus yielding wave vectors of the reciprocal lattice
	\todo{since we are dealing with the reciprocal lattice, we don't use G, right? We still use K?}

	\begin{equation} \label{eq:U(r)}
		U(\mathbf{r})
		=
		\sum_{\mathbf{K}}
		\mathbf{U_K}
		e^{i \mathbf{K} \cdot \mathbf{r}}
	\end{equation}

	Where the index over $\mathbf{K}$ indicates summation over all reciprocal lattice vectors.

	The Fourier coefficients $\mathbf{U_K}$ are related to $U(\mathbf{r})$ \todo{A\&M references appendix D here. Let's explain that a bit more.}

	\begin{equation} \label{eq:U_k}
		\mathbf{U_K}
		=
		\frac{1}{v}
		\int_{cell}
		d \mathbf{r}
		e^{-i \mathbf{K} \cdot \mathbf{r}}
		U(\mathbf{r})
	\end{equation}

	The potential energy is allowed to be referenced at any convenient value, so we require the spatial average $\mathbf{U_0}$ of the potential over a primitive cell to vanish:

	\begin{equation} \label{eq:U_0}
		\mathbf{U_0}
		=
		\frac{1}{v}
		\int_{cell}
		d \mathbf{r}
		U(\mathbf{r})
		= 0
	\end{equation}

	Note that because the potential $U(\mathbf{r})$ is real, it follows from eq. (\ref{eq:U_k}) that the Fourier coefficients satisfy

	\begin{equation} \label{eq:U_conj}
		\mathbf{U_{-K}} = \mathbf{U_{K}^*}
	\end{equation}

	If we assume inversion symmetry (solely for the sake of notation) \todo{really? can we explain here?} \todo{also how would we prove this without inversion symmetry?}
	so that, for a suitable choice of origin, $U(\mathbf{r}) = U(-\mathbf{r})$, then (\ref{eq:U_k}) implies that $\mathbf{U_K}$ is real. This can be seen by using Euler's formula; $\sin(0) = 0$. This allows us to write \todo{how does this help us?} the Schrödinger equation, eq. (\ref{eq:schrod}) \todo{don't understand what $\mathbf{c_q} is$}

	\begin{equation} \label{eq:hamiltonian}
		\frac{p^2}{2m} \psi
		=
		-\frac{\hbar^2}{2m}
		\boldsymbol{\nabla}^2 \psi
		=
		\sum_{q} \frac{\hbar^2}{2m}
		q^2 \mathbf{c_q} 
		e^{i\mathbf{q} \cdot \mathbf{r}}
	\end{equation}

	Making the substitution $\mathbf{K+q=q'}$ and the fact that eq. (\ref{eq:wave_vector}) is equivalent regardless of whether summing over $\mathbf{q}$ or $\mathbf{q'}$ \todo{explain this because it is not actually a sum over $\mathbf{q}$}.

	Using (\ref{eq:waves_q}) and (\ref{eq:U(r)}), the term in the potential energy can be written 
	\begin{align} \label{eq:U_simpl1}
		\mathbf{U} \psi
		&=
		\left( \sum_{K}
		e^{i \mathbf{K} \cdot \mathbf{r}} \right)
		\left( \sum_{K} \mathbf{c_q}
		e^{i \mathbf{q} \cdot \mathbf{r}} \right) \\ \nonumber
		&=		
		\sum_{\mathbf{Kq}}
		\mathbf{U_Kc_q}
		e^{i(\mathbf{K+q}) \cdot \mathbf{r}} \\
		&=
		\sum_{\mathbf{Kq'}}
		\boldsymbol{U}_{{\mathbf{K}{q' {-K}}}}
		e^{i(\mathbf{q'}) \cdot \mathbf{r}}
	\end{align}

	\todo{I do not get how we can just eliminate the $K$ in the exponent without doing anything to the $U_K$}
	\todo{I also don't understand what it means to have an eigenvalue of $\mathbf{q'-K}$}

	We can then change the summation indices in eq. (\ref{eq:U_simpl1}) from $\mathbf{K}$ and $\mathbf{q'}$ to $\mathbf{K'}$ and $\mathbf{q}$ so that the Schrödinger equation becomes

	\begin{equation} \label{eq:SE_1}
		\sum_{q} e^{i\mathbf{q}\cdot\mathbf{r}}
		\left\{
		\left(
		\frac{\hbar^2}{2m} q^2 - \varepsilon
		\right) \boldsymbol{c}_{\mathbf{q}}
		+ \sum_{\mathbf{K'}} \boldsymbol{U}_{\mathbf{K'}}
		\cdot \boldsymbol{c}_\mathbf{q {-K'}}
		\right\}
		= 0
	\end{equation}

	\todo{can someone explain explicitly how eq{\ref{eq:hamiltonian}} was used in eq{\ref{eq:SE_1}}}?

	The plane waves satisfying the Born-von Karman boundary condition are orthogonal \todo{can we explicitly say why?}, so the coefficients corresponding to separate terms in (\ref{eq:SE_1}) vanish \todo{how does this work mathematically?}

	therefore for all allowed wave vectors $\mathbf{q}$,

	\begin{equation} \label{eq:q_1}
		\left(
		\frac{\hbar^2}{2m} q^2
		- \varepsilon
		\right) \boldsymbol{c}_{\mathbf{q}}
		+ \sum_{K'} \boldsymbol{U}_\mathbf{K'}
		\cdot \boldsymbol{c}_\mathbf{q{-K'}}
		= 0
	\end{equation}

	We can write $\mathbf{q=k-K}$ where $\mathbf{K}$ is a reciprocal lattice vector chosen so that $\mathbf{k}$ lies in the first Brillouin zone. \todo{Please explain how this comes about}. Eq. (\ref{eq:q_1}) becomes

	\begin{equation} \label{eq:q_2}
		\left(
		\frac{\hbar^2}{2m}
		(\mathbf{k} - \mathbf{K})^2 - \varepsilon
		\right) \boldsymbol{c}_{\mathbf{k {-K}}}
		+ \sum_{K'} \boldsymbol{U}_\mathbf{K'}
		\boldsymbol{c}_\mathbf{k-K-K'}
		= 0
	\end{equation}

	If we make the change of variables \todo{I don't understand how we can make this change of variables?} $\mathbf{K'}\rightarrow \mathbf{K'-K}$,

	\begin{equation} \label{eq:SE_momentum}
		\left(
		\frac{\hbar^2}{2m}
		(\mathbf{k-\mathbf{K}})^2 - \varepsilon
		\right) \boldsymbol{c}_{\mathbf{k-K}}
		+ \sum_{K'} \boldsymbol{U}_{\mathbf{K'-K}}
		\boldsymbol{c}_{\mathbf{k-K'}}
		= 0
	\end{equation}

	These are simplifications of the Schrödinger equation in momentum space. \todo{what do they mean by momentum space if it is not k space?} The potential, $\boldsymbol{U}_\mathbf{k}$ is nonvanishing only when $\boldsymbol{k}$ is a vector of the reciprocal lattice.

	For fixed $\mathbf{k}$ in the first Brillouin zone, the set of equations \ref{eq:SE_momentum} for all reciprocal lattice vectors $\mathbf{K}$ couples only those coefficients $\boldsymbol{c}_\mathbf{k}$, $\boldsymbol{c}_\mathbf{k-K}$, $\boldsymbol{c}_\mathbf{k-K'}$, ... whose wave vectors differ from $\mathbf{k}$ by a reciprocal lattice vector. As a result, the original problem has been separated into $N$ independent problems: one for each allowed value of $\mathbf{k}$ in the first Brillouin zone. Each problem has solutions that are superpositions of plane waves containing only the wave vector $\mathbf{k}$ and wave vectors differing from $\mathbf{k}$ by a reciprocal lattice vector. \todo{the above was copied. I don't know how to change it. If we could add something that would be nice}

	Using the complex exponentials as a basis (\ref{eq:waves_q}, the wave function will be of the form

	\begin{equation} \label{eq:waves_k}
		\psi_{\mathbf{k}}
		= \sum_{\boldsymbol{K}}
		\boldsymbol{c}_{\mathbf{k-K}}
		e^{i(\mathbf{k-K}) \cdot \mathbf{r}}
	\end{equation}

	We can therefore write the wave function as \todo{I don't get how eq. \ref{eq:waves_q}}

	\begin{equation} \label{eq:psi_proof}
		\psi_{\mathbf{k}}(\mathbf{r})
		= e^{i\mathbf{k} \cdot \mathbf{r}}
		\left(\sum_{K} \boldsymbol{c}_{\mathbf{k-K}}
		e^{-i\mathbf{K\cdot r}}
		\right)
	\end{equation}

	This is of the Bloch form (\ref{eq:bloch_theorem}) with the periodic function $u(\mathbf{r})$ given by

	\begin{equation} \label{eq:u_r}
		u(\mathbf{r}) = 
		\sum_{\mathbf{K}} 
		\boldsymbol{c}_{\mathbf{k-K}}
		e^{-i\mathbf{K \cdot r}}
	\end{equation}

	There are infinitely many solutions to the infinite set of equations (\ref{eq:SE_momentum}) for a given $\mathbf{k}$, classified by the band index $n$.

\section{Remarks about Bloch's Theorem}
	\subsection{Crystal momentum}
		Bloch's theorem introduces a wave vector $\mathbf{k}$ in eq. (\ref{eq:bloch_theorem}) and eq. (\ref{eq:k_with_bi}), which plays the same fundamental role in the general problem of motion in a periodic potential that the free electron wave vector $\mathbf{k}$ plays in the Sommerfeld theory. \todo{What is this fundamental role?!?}

		However, the free electron wave vector is $\mathbf{p}/\hbar$, where $\mathbf{p}$ is the momentum of the electron. However, the Bloch vector $\mathbf{k}$ is not proportional to the electronic momentum; the Hamiltonian does not have complete translational invariance in the presence of a nonconstant potential, so its eigenstates will not be simulataneous eigenstates of the momentum operator. \todo{Can we show why the Hamiltonian doesn't have translational invariance?} When we apply the momentum operator, $\mathbf{p}=(\hbar/i)\boldsymbol{\nabla}$ on $\psi_{n\mathbf{k}}$ we obtain

		\begin{align} \label{momentum}
			\frac{\hbar}{i}\boldsymbol{\nabla}
			\psi_{n\mathbf{k}}
			&= \frac{\hbar}{i}\boldsymbol{\nabla}
			\left(
			e^{i\mathbf{k \cdot r}}
			u_{n\mathbf{k}}(r)
			\right) \\
			&= \hbar \mathbf{k} \psi_{n\mathbf{k}}
			+ e^{i\mathbf{k \cdot r}}
			\frac{\hbar}{i} \boldsymbol{\nabla}
			u_{n\mathbf{k}} (r)
		\end{align}

		which is not a constant times $\psi_{n\mathbf{k}}$; i.e. $\psi_{n\mathbf{k}}$ is not a momentum eigenstate. \todo{in which case would it be? when the gradient of u is zero? when is that?}

		However, $\hbar \mathbf{k}$ is known as the crystal momentum \todo{expand upon this- see chapter 12- response of Bloch e- to EM field} of the electron. $\mathbf{k}$ is a quantum number characteristic of the translational symmetry of a periodic potential, just as the momentum $\mathbf{p}$ is a quantum number characteristic of the fuller translational symmetry of free space. \todo{do not get this- explain}

	\subsection{First Brillouin zone}
		The Bloch wave vector $\mathbf{k}$ can be confined to the first Brillouin zone, or any other primitive cell of the reciprocal lattice. This is because any $\mathbf{k'}$ not in the first Brillouin zone can be written as

			\begin{equation} \label{k_prime}
				\mathbf{k'=k+K}
			\end{equation}

		where $\mathbf{K}$ is a reciprocal lattice vector and $\mathbf{k'}$ is not in the first zone. \todo{AM says k, not kprime here!!} Since $e^{i\mathbf{K \cdot R}}=1$ for any reciprocal lattice vector, if the Bloch form (\ref{eq:bloch_theorem_restate}) holds for $\boldsymbol{k'}$, it will also hold for $\boldsymbol{k}$.

	\subsection{The index $n$ and eigenvalues}
		The index $n$ appears in Bloch's theorem because for a given $\mathbf{k}$, there are many solutions to the Schrödinger equation (\ref{eq:schrod}). To prove this, we look at the solutions to (\ref{eq:schrod}) with the Bloch form

		\begin{equation*}
			\psi(\mathbf{r})
			= e^{i\mathbf{k\cdot r}}
			u(\mathbf{r})
		\end{equation*}

		where $\mathbf{k}$ is fixed and $u$ has the periodicity of the Bravais lattice. Substituting this into the Schrödinger equation, $u$ is determined by the eigenvalue problem

		\begin{align} \label{eq:eigenvalue}
			H_{\mathbf{k}} u_{\mathbf{k}} (\mathbf{r})
			&= \left( \frac{\hbar^2}{2m}
			\left(\frac{1}{i} \boldsymbol{\nabla}
			+ \mathbf{k} \right)^2
			+ U(\mathbf{r}) \right)
			u_\mathbf{k} (\mathbf{r}) \\
			&= \varepsilon_{\mathbf{k}} 
			u_\mathbf{k}(\mathbf{r})
		\end{align}

		with the boundary condition

		\begin{equation*}
			u_\mathbf{k} (\mathbf{r})
			= u_\mathbf{k} (\mathbf{r} + \mathbf{R})
		\end{equation*}

		Because of the boundary condition, we can regard (\ref{eq:eigenvalue}) restricted to a single primitive cell of the crystal. Because it is set in a fixed finite volume, we expect on general grounds to find an infinite family of solutions with discretely spaced eigenvalues, labeled with the band index $n$.

		According to (\ref{eq:eigenvalue}), the wave vector $\mathbf{k}$ appears only as a parameter in the Hamiltonian $H_{\mathbf{k}}$, which manes that the energy levels for a given $\mathbf{k}$ vary continuously as $\mathbf{k}$ varies. Therefore, we arrive at a description of the levels in a periodic potential in terms of a family of continuous functions $\varepsilon_{n} (\mathbf{k})$. \todo{i don't understand this. we should explain}

	\subsection{The index $n$ and band structure}
		If we allow $\mathbf{k}$ to range through all of $k$-space (instead of just the first Brillouin zone) \todo{is this correct?}, we can express all of the eigenstates and eigenvalues of a particular $n$ as being periodic functions of $\mathbf{k}$ in the reciprocal lattice:

		\begin{align} \label{eq:eigen_k_K}
			\psi_{n,\mathbf{k+K}} (\mathbf{r})
			&= \psi_{n\mathbf{k}} (\mathbf{r}), \\ \nonumber
			\varepsilon_{n, \mathbf{k+K}}
			&= \varepsilon_{n\mathbf{k}}.
		\end{align}


		The energy levels of an electron in a periodic potential can thus be expressed in terms of a family of continuous functions $\varepsilon_{n\mathbf{k}}$ (or $\varepsilon_{n}(\mathbf{k}$), each with the periodicity of the reciprocal lattice. This is referred to as the \emph{band structure} of the solid. \todo{so does this mean that we have wavefunctions for each energy, labeled by $n$, and these wavefunctions are functions of $\mathbf{k}$?}

		\todo{can we briefly explain what a "band" is, even though it is discussed in Chapter 10?}

		The levels $\varepsilon_n(\mathbf{k})$ lie in the band energies between the upper and lower bound energies since $\varepsilon_n(\mathbf{k})$ is periodic in $\mathbf{k}$ and continuous.
		\todo{I don't understand what the "upper and lower bounds" are. }

	\subsection{Non-vanishing mean velocity}
		It can be shown (see Appendix E of AM) \todo{cite this} that an electron in a level specified by band index $n$ and wave vector $\mathbf{k}$ has a nonvanishing mean velocity given by

		\begin{equation} \label{eq:mean_v}
			\mathbf{v}_n(\mathbf{k})
			= \frac{1}{\hbar}
			\boldsymbol{\nabla}_\mathbf{k}
			\varepsilon_n (\mathbf{k})
		\end{equation}

		\todo{explain a little more about where this comes from} This asserts that there are stationary (i.e. time-independent) levels for an electron in a periodic potential in which, in spite of the interaction of the electron with the fixed lattice of ions, it moves forever without any degradation of its mean velocity. This contrasts to the Drude model in which collisions were encounters between the electron and a static ion. \todo{Did the Drude model have that time constant, $\tau$, that prevented a mean velocity? Let's explain that here.}
		\todo{explain a little what happens in chapters 12 and 13 with the mean velocity}

\section{The Fermi Surface} \todo{heidi will rephrase the whole section}
	The ground state of $N$ \emph{free electrons} \todo{why do they mention that they are not distinguishing between the number of conduction electrons and the number of primitive cells?} is constructed by occupying all one-electron levels $\mathbf{k}$ with energies $\varepsilon (\mathbf{k}) = \hbar^2 k^2 / 2m$ less than $\varepsilon_F$, where $\varepsilon_F$ is determined by requiring the total number of one-electron levels with energies less than $\varepsilon_F$ to be equal to the total number of electrons. 
	The ground state of $N$ \emph{Bloch electrons} is similarly constructed, except that the one-electron levels are now labeled by quantum numbers $n$ and $\mathbf{k}$, $\varepsilon_F$ does not have the explicit free electron form \todo{what form is this?}, and $\mathbf{k}$ must be confined to a single primitive cell of the reciprocal lattice if each level is to be counted only once. When the lowest of these levels are filled by a specified number of electrons, two quite distinct types of configuration can result: \todo{refrase this and the items below}
	\begin{enumerate}
		\item A certain number of bands may be completely filled with all others remaining epty. The difference in energy between the highest occupied level and lowest unoccupied level (i.e. between the "top" of the highest occupied band and the "bottom" of the lowest empty band) is known as the \emph{band gap}. We shall find that solids with a band gap greatly in excess of $k_BT$ ($T$ near room temperature) are insulators. If the band gap is equal to the number of primitive cells in the crystal \todo{see page 136} and because each level can accommodate two electrons (one of each spin), a configuration with a band gap can arise (though it need not) only if the number of electrons per primitive cell is even. \todo{read chapter 9}
		\item A number of bands may be partially filled. When this occurs, the energy of the highest occupied level, the Fermi energy $\varepsilon_F$, lies within the energy range of one or more bands. For each partially filled band there will be a surface in $k$-space separating the occupied from the unoccupied levels. The set of all such surfaces is known as the \emph{Fermi surface} and is the generalization to Bloch electrons of the free electron Fermi sphere. The parts of the Fermi surface arising from individual partially filled bands are known as \emph{branches} of the Fermi surface. In many important cases, the Fermi surface is entirely within a single band, and generally it is found to lie within a fairly small number of bands. \todo{see chapter 15} A solid has metallic properties provided that a Fermi surface exists. \todo{see chapter 12}
	\end{enumerate}

	Analytically, the branch fo the Fermi surface in the $n$th band is that surface in $k$-space (if there is one) determined by
	\begin{equation} \label{eq:e_k}
		\varepsilon_n(\mathbf{k}) = \varepsilon_F
	\end{equation}

	Thus the Fermi surface is a constant energy surce (or set of constant energy surfaces) \todo{don't know why a set} in $k$-space, just as the more familiar equipotentials of electrostatic theory are constant energy surfaces in real space.
	If $\varepsilon_F$ is generally defined as the energy separating the highest occupied from the lowest unoccupied level, then it is not uniquely specified in a solid with an energy gap, since any energy in the gap meets this test. People nevertheless speak of ``\emph{the} Fermi energy'' of an intrinsic semiconductor. What they mean is the chemical potential, which is well defined at any nonzero temperature \todo{see appendix B}. As $T\rightarrow 0$, the chemical potential of a solid with an energy gap approaches the energy at the middle of the gap (page 575), and one sometimes finds it asserted that this is the ``Fermi energy'' of a solid with a gap. With either the correct (undetermined) or colloquial definition of $\varepsilon_F$, (\ref{eq:e_k}) asserts that solids with a gap have no Fermi surface.

	Since the $\varepsilon_n(\mathbf{k})$ are periodic in the recprocal lattice, the complete solution to (\ref{eq:e_k}) for each $n$ is a $k$-space surface with the periodicity of the reciprocal lattice. When a branch of the Fermi surface is represented by the full periodic structure, it is said to be described in a \emph{repeated zone scheme}. Often, however, it is preferable to take just enough of each branch of the Fermi surface so that every physically distinct level is represented by just one point of the surface. This is acheived by representing each branch by that portion of the full periodic surface contained within a single primitive cell of the reciprocal lattice. Such a representation is described as a \emph{reduced zone scheme}. The primitive cell chosen is often, but not always, the first Brillouin zone. \todo{see ch 9 and 15}
