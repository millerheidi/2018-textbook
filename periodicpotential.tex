%----------------------------------------------------------------------------------------
%	ELECTRON LEVELS IN A PERIODIC POTENTIAL
%----------------------------------------------------------------------------------------
\chapter{Electron Levels in a Periodic Potential}

\listoftodos

\section{Preface}

\todo{We should somehow rename this section}

\todo{I think we should organize it by learning objectives, introduction, main text, and summary, as the prof has outlined in the template.}

\todo{I think we should go through the notes from his slides to see if there is anything valuable there.}

\todo{We need to check the notation for $K$ and $G$ since we should be consistent with the notes and not A\&M}

\todo{we might want to write a summary of the main points and equations}

The writers' goals are:
\begin{enumerate}
	\item To give a clearer picture of the direction of the proof before finding the result
	\item To explain things that were not fully explained
	\item To add extra insight and applications to A\&M
\end{enumerate}

\todo{I think that the more sources we have, the better it looks.}

\begin{quotation}
	``When I started to think about it, I felt that the main problem was to explain how the electrons could sneak by all the ions in a metal so as to avoid a mean free path of the order of atomic distances. Such a distance was much too short to explain the observed resistances, which even demanded that the mean free path become longer and longer with decreasing temperature. But Heitler and London had already shown how electrons could jump between two atoms in a molecule to form a covalent bond, and the main difference between a molecule and a crystal was only that there were many more atoms in a periodic arrangement. To make my life easy, I began by considering wave functions in a one-dimensional periodic potential. By straight Fourier analysis I found to my delight the the wave differed from a a plane wave of free electron only by a periodic modulation.
	This was so simple that I didn’t think it could be much of a discovery, but when I showed it to Heisenberg he said right away, “That’s it.” Well, that wasn’t quite it yet, and my calculations were only completed in the summer when I wrote my thesis on 'The Quantum Mechanics of Electrons in Crystal Lattices.''
\end{quotation}
- Felix Bloch


These waves are now referred to as "Bloch Waves" or "Bloch States", and they are a fundamental building bloch of condensed matter physics. This was already known in mathematics as Floquet's Theorem, but Bloch discovered its immense implications for quantum mechanics.
Biographical Memoirs, Volume 64 \todo{cite properly}

\section{Introduction}
	Recall the free electron and independent electron approximations from the assumptions of the Sommerfeld Model:

	\begin{enumerate}
		\item \textbf{Free electron approximation.} The positive ions are there only to maintain charge neutrality. The interaction between ions and valence electrons is neglected except in boundary conditions.
		\item \textbf{Independent electron approximation.} The interactions between electrons are neglected.
	\end{enumerate}


	Although the free electron approximation provides insight into properties of metals, such as heat capacity, thermal conductivity, and electrical conductivity, it does not account for the differences between metals, semiconductors, and insulators, and it cannot account for positive values of the Hall coefficient. In addition, conduction electrons in the metal do not correctly correspond to the number of valence electrons in free atoms. The purpose of this chapter is to set the stage for understanding band structure.

\section{Periodic Potential}

	\todo{Find something similar to figure 8.1 and include it. See latex.tex for how to include the figure}

	In a perfect crystal, the electron potential looks the same at each point on the crystal lattice. This can be expressed mathematically as:
	\begin{equation} \label{eq:U_period}
		U(\mathbf{r} + \mathbf{R}) = U(\mathbf{r})
	\end{equation}
	where $\mathbf{R}$ is the Bravais lattice vector. The scale of periodicity of the potential, $U$ ($\sim$\num{d-8} cm), is the size of a typical de Broglie wavelength of an electron in the Sommerfeld free electron model, so quantum mechanics can describe the wavefunction with this same periodicity. 

	The electronic properties of a crystal can be examined with the Schrödinger equation while maintaining the \textit{independent electron approximation}:
	\begin{equation} \label{eq:schrod}
		H\psi=\left(-\frac{\hbar^2}{2m}\mathbf{\nabla}^2 + U(\mathbf{r})\right)\psi = \varepsilon\psi
	\end{equation}

	These electrons are known as \textit{Bloch electrons}, and they reduce to ``free electrons'' when the periodic potential, $U(\mathbf{r})$ is equal to zero.

	Bloch proved that the solutions to the Schrödinger equation for a periodic potential must be of the form:

	\begin{equation} \label{eq:bloch_theorem}
		\psi_{n\mathbf{k}}(\mathbf{r}) 
		= e^{i\mathbf{k} \cdot \mathbf{r}}
		u_{n\mathbf{k}}(r)
	\end{equation}

	where $u_{n\mathbf{k}}(\mathbf{r})$ has the period of the crystal lattice with $u_{n\mathbf{k}}(\mathbf{r}) = u_{n\mathbf{k}}(\mathbf{r} + \mathbf{T})$ \todo{We should explain where the n comes from.}
	Here, $\mathbf{T}$ is a translation vector of the lattice. This result (eq. \ref{eq:bloch_theorem}), expresses the Bloch theorem:
	\begin{quotation}
		The eigenstates of the wave equation can be chosen so that, for every $\mathbf{R}$ in the Bravais Lattice, each $\psi$ has an associated wave vector, $\mathbf{k}$, such that:
	\end{quotation}

	\begin{equation} \label{eq:bloch_theorem_restate}
		\psi(\mathbf{r} + \mathbf{R})
		= e^{i\mathbf{k} \cdot \mathbf{R}}
		\psi(\mathbf{r})
	\end{equation}


	Ashcroft and Mermin provide two proofs of Bloch's Theorem, which we will reproduce below.

\section{Quantum Mechanical proof of Bloch's Theorem}
	This proof is valid even when $\psi_{\mathbf{k}}$ is degenerate, \todo{put this somewhere else. Also check about the other proof} i.e. when there are other wavefunctions with the same energy and wavevector as $\psi_{\mathbf{k}}$.

	For each Bravais lattice vector $\mathbf{R}$, let $T_{\mathbf{R}}$ be the translation operator, which, when operating on any function $f(\mathbf{r})$, shifts the argument by $\mathbf{R}$:

	\begin{equation} \label{eq:translation}
		T_{\mathbf{R}} f(\mathbf{r})
		= f(\mathbf{r} + \mathbf{R})
	\end{equation}

	Due to the periodicicity of the lattice, we can show that the translation operator commutes with the Hamiltonian:

	\begin{equation} \label{eq:trans_commute1}
		T_{\mathbf{R}} H \psi
		= H \psi(\mathbf{r} + \mathbf{R})
		= H(\mathbf{r}) \psi(\mathbf{r} + \mathbf{R})
		= HT_{R} \psi
	\end{equation}

	Note that the application of two successive translations does not depend on the order in which they are applied, since for any $\psi(\mathbf{r})$

	\begin{equation} \label{eq:trans_commute2}
		T_{\mathbf{R}} T_{\mathbf{R'}} \psi(\mathbf{r})
		= T_{\mathbf{R'}} T_{\mathbf{R}} \psi(\mathbf{r})
		= \psi(\mathbf{r} + \mathbf{R} + \mathbf{R'})
	\end{equation}

	Therefore
	\begin{equation} \label{eq:trans_addition}
		T_{\mathbf{R}} T_{\mathbf{R'}}
		= T_{\mathbf{R'}} T_{\mathbf{R}}
		= T_{\mathbf{R + R'}}
	\end{equation}

	Because eq. (\ref{eq:trans_commute1}) holds identically for any function $\psi$, $T_{\mathbf{R}}$ and $H$ commute:

	\begin{equation} \label{eq:trans_commute_relation}
		T_{\mathbf{R}} H = H T_{\mathbf{R}}
	\end{equation}

	A theorem of quantum mechanics \todo{maybe we should say which theorem. see AM subnote on page 134.} allows us to choose eigenstates of $H$ that are simultaneous eigenstates of all the $T_{\mathbf{R}}$:

	\begin{eqnarray} 
		H\psi &=& \varepsilon\psi \nonumber \\
		T_{R} \psi &=& c(\mathbf{R}) \psi \label{eq:eigenstates}
	\end{eqnarray}

	The eigenvalues $c(\mathbf{R})$ of the translation operators are related in the following way:
	\begin{equation} \label{eq:c_relation}
		c(\mathbf{R} + \mathbf{R'}) 
		= c(\mathbf{R}) c(\mathbf{R}')
	\end{equation}

	This can be seen from eq. (\ref{eq:trans_addition}):
	\begin{equation} \label{eq:derive_crelation1}
		T_{\mathbf{R}'} T_{\mathbf{R}} \psi
		= c(\mathbf{R}) T_{\mathbf{R}} \psi
		= c(\mathbf{R}) c(\mathbf{R}')
	\end{equation}

	Eq. (\ref{eq:trans_addition}) can \emph{also} be used to formulate the eigenvalues as such:
	\begin{equation}
		T_{\mathbf{R}'} T_{\mathbf{R}} \psi
		= T_{\mathbf{R} + \mathbf{R'}} \psi
		= c(\mathbf{R} + \mathbf{R'}) 
	\end{equation}

	In order to satisfy Bloch's theorem, in the form of eq. (\ref{eq:bloch_theorem_restate}), we can come up with an expression for the exponential, $e^{i\mathbf{k} \mathbf{R}}$ in terms of the eigenvalue, $c(\mathbf{R})$ through the following derivation.

	We first write an expression for $\mathbf{R}$ as the Bravais lattice vector given by
	\begin{equation} \label{eq:r_express}
		\mathbf{R} = n_{1} \mathbf{a}_{1}
		+ n_{2} \mathbf{a}_{2}
		+ n_{3} \mathbf{a}_{3}
	\end{equation}

	where $\mathbf{a_{i}}$ are the three primitive vectors of the Bravais lattice, which can be captured by $\mathbf{c}$: \todo{Can someone please explain how we can express $c$ this way because I do not get it! Thanks.}

	\begin{equation} \label{eq:c_exp}
			c(\mathbf{a}_{i}) = e^{2 \pi i x_{i}}
	\end{equation}

	by a suitable choice \todo{yeah I don't know what ``suitable'' means. Please explain} of $x_i$. From successive applications of \ref{eq:c_relation} that

	\begin{equation} \label{eq:c_mult}
		c(\mathbf{R})
		= c(\mathbf{a}_1)^{n_1}
		c(\mathbf{a}_2)^{n_2}
		c(\mathbf{a}_3)^{n_3}
	\end{equation}

	and since
	\begin{equation} \label{eq:k_with_bi}
		\mathbf{k} = x_1 \mathbf{b}_1
		+ x_2\mathbf{b}_2
		+ x_3 \mathbf{b}_3
	\end{equation}

	where $b_i$ are the reciprocal lattice vectors satisfying
	\begin{equation} \label{eq:recip_b}
		\mathbf{b_i \cdot a_j}
		= 2 \pi \delta_{ij}
	\end{equation}

	this means that
	\begin{equation} \label{eq:c(R)}
		c(\mathbf{R}) 
		= e^{i \mathbf{k} \cdot \mathbf{R}}
	\end{equation}

	We can thus choose the eigenstates $\psi$ of $H$ so that for every Bravais lattice vector $\mathbf{R}$, we can obtain Bloch's theorem:
	\begin{equation} \label{eq:bloch_proof_qm}
		T_{\mathbf{R}} \psi
		= \psi(\mathbf{r} + \mathbf{R}) \psi
		= e^{i \mathbf{k} \cdot \mathbf{R}}
	\psi(\mathbf{r})
	\end{equation}
	\todo{I think we can add some things from the PDF I sent. We should find out what sources the PDF used and include that}

\section{Form of the Bloch Wave Vector}
	\todo{Note here that I renamed this section from the BVK Boundary Conditions.}
	To demonstrate that the wave vector $\mathbf{k}$ must be real, we must impose the appropriate boundary conditions. The Born-Von Karman boundary condition was originally presented earlier as
	\begin{eqnarray} \label{eq:bvk_ch2}
		\psi(x,y,z+L) &=& \psi(x,y,z), \\ \nonumber
		\psi(x,y+L,z) &=& \psi(x,y,z), \\
		\psi(x+L,y,z) &=& \psi(x,y,z) \nonumber
	\end{eqnarray}

	but these boundary conditions apply to a cubic Bravais lattice with side L that is an integral multiple of the lattice constant, $a$. A general form of eq. (\ref{eq:bvk_ch2}) is 
	\begin{alignat}{2}\label{eq:bvk}
		\psi(\mathbf{r} + N_i \mathbf{a}_i)
		&= \psi(\mathbf{r}),
		&\quad
		i &= 1,2,3
	\end{alignat}

	The $\mathbf{a}_i$ are the primitive vectors, and the $N_i$ are all integers of order $N^{1/3}$, where $N=N_1N_2N_3$ is the total number of primitive cells in the crystal.

	Applying Bloch's theorem, eq. (\ref{eq:bloch_theorem_restate}) to the boundary condition, eq. (\ref{eq:bvk}), \todo{can someone please explain where the n comes from?} we obtain

	\begin{alignat}{2}\label{eq:bloch_applied}
		\psi_{n\mathbf{k}}
		(\mathbf{r} + N_i \mathbf{a}_i)
		&= 
		e^{i N_i \mathbf{k} \cdot \mathbf{a}_i}
		\psi(\mathbf{r}),
		&\quad
		i &= 1,2,3
	\end{alignat}

	which requires that

	\begin{equation} \label{eq:requirement}
		e^{2 \pi i N_i x_i} = 1
	\end{equation}

	meaning
	\begin{alignat}{2} \label{eq:xi}
		x_i &= \frac{m_i}{N_i},
		& \quad
		m_i\text{ integral}
	\end{alignat}

	\todo{I think I need to edit this section a bit more because it is just copied- there are no additional explanations}

	Therefore the general form for allowed Bloch wave vectors is

	\begin{alignat}{3} \label{eq:wave_vector}
		\mathbf{k} =
		\sum_{i=1}^{3} \frac{m_i}{N_i}
		\mathbf{b_i}
		,
		&\quad 
		&m_i\text{ integral}
	\end{alignat}

	It follows from eq. (\ref{eq:wave_vector}) that the volume $\Delta\mathbf{k}$ of $k$-space per allowed value of $\mathbf{k}$ is just the volume of the little parallelepiped with edges $\mathbf{b}_i/N_i$:

	\begin{equation} \label{eq:deltak}
		\Delta{k}
		= 
		\frac{\mathbf{b_1}}{N_1}
		\cdot
		\left(
		\frac{\mathbf{b_2}}{N_2}
		\times
		\frac{\mathbf{b_3}}{N_3}
		\right)
		=
		\frac{1}{N}
		\mathbf{b_1}
		\cdot
		\left(
		\mathbf{b_2}
		\times
		\mathbf{b_3}
		\right)
	\end{equation}

	Since $\mathbf{b_1} \cdot (\mathbf{b_2} \times \mathbf{b_3}$ is the volume of a reciprocal lattice primitive cell, eq. (\ref{eq:deltak}) asserts that \emph{the number of allowed wave vectors in a primitive cell of the reciprocal lattice is equal to the number of sites in the crystal}.

	The volume of a reciprocal lattice primitive cell is $(2π)^3/v$, where $v=V/N$ is the volume of a direct lattice primitive cell, so eq. (\ref{eq:deltak}) can be written in the alternative form:
	\begin{equation} \label{eq:freedeltak}
		\Delta \mathbf{k}
		=
		\frac{(2 \pi)^3}{V}
	\end{equation}

	which is the same as the free electron volume.
	\todo{expand upon this}

\section{Proof 2 of Bloch's Theorem}
	The first proof of Bloch's Theorem allowed us to formulate the expression for the Bloch wave vector as a function of the reciprocal lattice vectors. \todo{correct?} This second proof is useful in later discussions pertaining to band energies.

	Using eq. (\ref{eq:wave_vector}), one can always expand any function obeying the Born-von Karman boundary condition, (\ref{eq:bvk}) in the set of all plane waves that satisfy the boundary condition, resulting in wave vectors with the form:

	\begin{equation} \label{eq:waves_q}
		\psi(\mathbf{r})
		=
		\sum_{q}
		\mathbf{c_q}
		e^{i\mathbf{q} \cdot \mathbf{r}}
	\end{equation}

	Because the potential $U(\mathbf{r})$ is periodic in the lattice, its \emph{plane wave expansion} will only contain plane waves with the periodicity of the lattice, thus yielding wave vectors of the reciprocal lattice
	\todo{since we are dealing with the reciprocal lattice, we don't use G, right? We still use K?}

	\begin{equation} \label{eq:U(r)}
		U(\mathbf{r})
		=
		\sum_{\mathbf{K}}
		\mathbf{U_K}
		e^{i \mathbf{K} \cdot \mathbf{r}}
	\end{equation}

	Where the index over $\mathbf{K}$ indicates summation over all reciprocal lattice vectors.

	The Fourier coefficients $\mathbf{U_K}$ are related to $U(\mathbf{r})$ \todo{A\&M references appendix D here. Let's explain that a bit more.}

	\begin{equation} \label{eq:U_k}
		\mathbf{U_K}
		=
		\frac{1}{v}
		\int_{cell}
		d \mathbf{r}
	\end{equation}















