%-----------------------------------------------------------------------------------------------
%  macros-2018.tex
%  Alastair McLean
%  2018 01 04
%-----------------------------------------------------------------------------------------------


%-----------------------------------------------------------------------------------------------
%  MACROS FOR SETTING UP PROBLEMS
%  Define locationofproblems in the source file
%-----------------------------------------------------------------------------------------------
\newcommand{\setupproblem}[3]{\noindent\textbf{Problem #1} \hfill \emph{#2}\\
\input{\locationofproblems/#3}}

\newcommand{\setupproblemcaps}[3]{\caps{Problem #1} \hfill \emph{#2}\\
\input{\locationofproblems/#3}}


%-----------------------------------------------------------------------------------------------
%  MARGINNOTE
%-----------------------------------------------------------------------------------------------
\newcommand{\marginnote}[1]{\marginpar{\small #1}}


%-----------------------------------------------------------------------------------------------
%  MINIPAGE
%-----------------------------------------------------------------------------------------------
\newcommand{\minipages}[6]{
\hspace{#1\linewidth}
\begin{minipage}[c]{#2\linewidth}
{#5}
\end{minipage}
\hspace{#3\linewidth}
\begin{minipage}[c]{#4\linewidth}
{#6}
\end{minipage}
}


%-----------------------------------------------------------------------------------------------
%  SHIFTRIGHT
%-----------------------------------------------------------------------------------------------
\newcommand{\shiftright}[3]{
\hspace{#1\linewidth}
\begin{minipage}[c]{#2\linewidth}
{#3}
\end{minipage}
}


%-----------------------------------------------------------------------------------------------
%  QUANTUM MECHANICS
%-----------------------------------------------------------------------------------------------
\newcommand{\blank}[1]{\placefigure{blank.pdf}{#1}} 
\newcommand{\curlybrackets}[1]{\{{#1}\}} 
\newcommand{\bra}[1]{\langle {#1}|} 
\newcommand{\braket}[2]{\langle {#1}|{#2} \rangle} 
\newcommand{\braOket}[3]{\langle {#1}|{#2}|{#3} \rangle} 
\newcommand{\commutator}[2]{[ {#1},{#2} ]} 
\newcommand{\expectation}[1]{\langle {#1} \rangle} 
\newcommand{\innerproduct}[2]{( {#1}|{#2} )} 
\newcommand{\ket}[1]{| {#1} \rangle} 
\newcommand{\momentumoperator}{-i\hbar\partial_x}
\newcommand{\QM}{Quantum Mechanics}
\newcommand{\Hamiltonian}{\hat H}
\newcommand{\ashcroft}{Ashcroft and Mermin}
\newcommand{\ashmer}{\textbf{am}}


%-----------------------------------------------------------------------------------------------
%  ASHCROFT MERMIN
%-----------------------------------------------------------------------------------------------
\newcommand{\ashcroftmermin}[1]{{\textbf{am} #1}} 

%-----------------------------------------------------------------------------------------------
%  COLORS BASIC
%-----------------------------------------------------------------------------------------------
\newcommand{\blue}[1]{{\color{blue}#1}}
\newcommand{\red}[1]{{\color{red}#1}}
\newcommand{\yellow}[1]{{\color{yellow}#1}}

%-----------------------------------------------------------------------------------------------
%  COLORS DVIPS
%-----------------------------------------------------------------------------------------------
\newcommand{\dandelion}[1]{{\color{Dandelion}#1}}
\newcommand{\emerald}[1]{{\color{Emerald}#1}}
\newcommand{\gray}[1]{{\color{Gray}#1}}

%-----------------------------------------------------------------------------------------------
%  COLORS SVGNAMES
%-----------------------------------------------------------------------------------------------
\newcommand{\aqua}[1]{{\color{Aqua}#1}}
\newcommand{\aquamarine}[1]{{\color{Aquamarine}#1}}
\newcommand{\blueviolet}[1]{{\color{BlueViolet}#1}}
\newcommand{\cadetblue}[1]{{\color{CadetBlue}#1}}
\newcommand{\crimson}[1]{{\color{Crimson}#1}}
\newcommand{\darkblue}[1]{{\color{DarkBlue}#1}}
\newcommand{\darkgray}[1]{{\color{DarkGray}#1}}
\newcommand{\darkgrey}[1]{{\color{DarkGrey}#1}}
\newcommand{\darkmagenta}[1]{{\color{DarkMagenta}#1}}
\newcommand{\darkolivegreen}[1]{{\color{DarkOliveGreen}#1}}
\newcommand{\darkorange}[1]{{\color{DarkOrange}#1}}
\newcommand{\darkturquoise}[1]{{\color{DarkTurquoise}#1}}
\newcommand{\deeppink}[1]{{\color{DeepPink}#1}}
\newcommand{\dimgray}[1]{{\color{DimGray}#1}}
\newcommand{\dimgrey}[1]{{\color{DimGrey}#1}}
\newcommand{\dodgerblue}[1]{{\color{DodgerBlue}#1}}
\newcommand{\firebrick}[1]{{\color{FireBrick}#1}}
\newcommand{\lightslategrey}[1]{{\color{LightSlateGrey}#1}}
\newcommand{\midnightblue}[1]{{\color{MidnightBlue}#1}}
\newcommand{\navyblue}[1]{{\color{NavyBlue}#1}}
\newcommand{\orange}[1]{{\color{Orange}#1}}

%-----------------------------------------------------------------------------------------------
%  SPECIAL COLORS
%-----------------------------------------------------------------------------------------------
\newcommand{\colorone}[1]{{\color{DarkBlue}#1}}
\newcommand{\colortwo}[1]{{\color{DarkOrange}#1}}
\newcommand{\colorthree}[1]{{\color{FireBrick}#1}}
\newcommand{\colorfour}[1]{{\color{DarkOliveGreen}#1}}

\newcommand{\colorkeyword}[1]{{\color{Teal}#1}}
%\newcommand{\colorkeyword}[1]{{\color{MidnightBlue}#1}}
\newcommand{\colorimportant}[1]{{\color{DarkOrange}#1}}
\newcommand{\colordefinition}[1]{{\color{DarkOrange}#1}}
\newcommand{\colorperson}[1]{{\color{MidnightBlue}#1}}
\newcommand{\colorname}[1]{{\color{MidnightBlue}#1}}

\newcommand{\highlight}{\hl}


% INTEGRALS
\newcommand{\integral}[4]{\int_{#1}^{#2}{#3}{#4}} 
\newcommand{\infiniteintegral}[2]{\int_{-\infty}^{\infty}{#1}{#2}} 
 
% TABLES
\newcommand{\e}{\\ \hline}


%-----------------------------------------------------------------------------------------------
%  QUOTATION MARKS
%-----------------------------------------------------------------------------------------------
\newcommand{\quotationmarks}[1]{{\lq\lq{#1}\rq\rq}{}}


%-----------------------------------------------------------------------------------------------
%  FIGURES
%-----------------------------------------------------------------------------------------------
\newcommand{\placefigure}[2]{\centerline{\includegraphics[width=#2 \columnwidth]{#1}}} 
\newcommand{\placefigurenocenter}[2]{{\includegraphics[width=#2 \columnwidth]{#1}}} 
\newcommand{\placeframedfigure}[2]{\begin{framed}\centerline{\includegraphics[width=#2 \columnwidth]{#1}}\end{framed}} 

\newcommand{\placefigurewithcaption}[3]{
\begin{figure}[htbp]
\begin{center}
\includegraphics[width=#2 \columnwidth]{#1}
\caption{#3}
\label{default}
\end{center}
\end{figure}
} 


\newcommand{\placefigurewithcaptionandlabel}[4]{
\begin{figure}[htbp]
\begin{center}
\includegraphics[width=#2 \columnwidth]{#1}
\caption{#3}
\label{#4}
\end{center}
\end{figure}
} 


%-----------------------------------------------------------------------------------------------
%  CRYSTAL STRUCTURES 
%-----------------------------------------------------------------------------------------------
\newcommand{\fcc}{fcc}
\newcommand{\bcc}{bcc}
\newcommand{\hcp}{hcp}
\newcommand{\dhcp}{dhcp}
\newcommand{\simplecubic}{sc}
\newcommand{\simplehexagonal}{sh}



%-----------------------------------------------------------------------------------------------
%  SHORTHAND
%-----------------------------------------------------------------------------------------------
\newcommand{\abinitio}{{\it ab inito}}
\newcommand{\boldit}[1]{{\bf #1}{}}
\newcommand{\boldfont}[1]{{\bf #1}{}}
\newcommand{\bsqrtthree}{{Si(111)-B($\sqrt 3{\times}\sqrt 3$)R30$^{\circ}$}}
\newcommand{\sqrtthree}{{($\sqrt 3{\times}\sqrt 3$)R30$^{\circ}$}}
\newcommand{\bsqrtthreeshort}{B$\sqrt 3$}
\newcommand{\centeredtypewriter}[1]{\centerline{\texttt{ #1}{}}}
\newcommand{\chemicalpotential}{\mu}
\newcommand{\dt}{\frac{d}{dt}}
\newcommand{\electronegativity}{\chi}
\newcommand{\emphasise}[1]{{\em #1}{}}
\newcommand{\emphasize}[1]{{\em #1}{}}
\newcommand{\epsilonnaught}{\varepsilon_\circ}
\newcommand{\etal}{{\it et al.}}
\newcommand{\eom}{{EOM}}
\newcommand{\equationshort}{{Eq.}}
\newcommand{\fancy}[1]{{\em #1}{}}
\newcommand{\fbd}{FBD}
\newcommand{\hardness}{\eta}
\newcommand{\infinity}{\infty}
\newcommand{\kineticenergy}{\frac{1}{2}mv^2}
\newcommand{\xa}{x$_a$}
\newcommand{\xb}{x$_b$}
\newcommand{\mone}{m$_1$}
\newcommand{\mtwo}{m$_2$}
\newcommand{\mthree}{m$_3$}
\newcommand{\mfour}{m$_4$}
\newcommand{\mfive}{m$_5$}
\newcommand{\msix}{m$_6$}
\newcommand{\mseven}{m$_7$}
\newcommand{\meight}{m$_8$}
\newcommand{\mnine}{m$_9$}
\newcommand{\ma}{m$_a$}
\newcommand{\mb}{m$_b$}
\newcommand{\mc}{m$_c$}
\newcommand{\md}{m$_d$}
\newcommand{\me}{m$_e$}
\newcommand{\mf}{m$_f$}
\newcommand{\mg}{m$_g$}
\newcommand{\mh}{m$_h$}
\newcommand{\mi}{m$_i$}
\newcommand{\opamp}{Op-Amp}
\newcommand{\ode}{ODE}
\newcommand{\omeganaught}{$\omega_\circ$}
\newcommand{\proportional}{\propto}
\newcommand{\question}{\textbf {Question}: }
\newcommand{\softness}{\sigma}
%\newcommand{\si}{Si(111)-(7$\times$7)}
\newcommand{\shm}{SHM}
\newcommand{\sevenbyseven}{Si(111)-(7$\times$7)}
\newcommand{\oneby}{1$\times$1}
\newcommand{\twoby}{2$\times$2}
\newcommand{\threeby}{3$\times$3}
\newcommand{\fourby}{4$\times$4}
\newcommand{\fiveby}{5$\times$5}
\newcommand{\sixby}{6$\times$6}
\newcommand{\sevenby}{7$\times$7}
\newcommand{\eightby}{8$\times$8}
\newcommand{\nineby}{9$\times$9}
\newcommand{\theanswer}{\emph {Answer}:}
\newcommand{\typewriter}[1]{\texttt{ #1}{}}
\newcommand{\under}{\underline}
\newcommand{\wavefunction}{WF}

%-----------------------------------------------------------------------------------------------
%  FRACTIONS
%-----------------------------------------------------------------------------------------------
\newcommand{\half}{\frac{1}{2}}
\newcommand{\onehalf}{\frac{1}{2}}
\newcommand{\threehalves}{\frac{3}{2}}
\newcommand{\third}{\frac{1}{3}}
\newcommand{\onethird}{\frac{1}{3}}
\newcommand{\twothirds}{\frac{2}{3}}
\newcommand{\quarter}{\frac{1}{4}}
\newcommand{\onequarter}{\frac{1}{4}}
\newcommand{\onefifth}{\frac{1}{5}}
\newcommand{\onesixth}{\frac{1}{6}}
\newcommand{\oneseventh}{\frac{1}{7}}
\newcommand{\oneeigth}{\frac{1}{8}}
\newcommand{\onenineth}{\frac{1}{9}}
\newcommand{\onetenth}{\frac{1}{10}}


%-----------------------------------------------------------------------------------------------
%  EASY TO REMEMBER SPACES
%-----------------------------------------------------------------------------------------------
\newcommand{\onespace}{\vspace{1em}}
\newcommand{\twospace}{\vspace{2em}}
\newcommand{\threespace}{\vspace{3em}}
\newcommand{\fourspace}{\vspace{4em}}
\newcommand{\fivespace}{\vspace{5em}}
\newcommand{\sixspace}{\vspace{6em}}
\newcommand{\sevenspace}{\vspace{7em}}
\newcommand{\eightspace}{\vspace{8em}}
\newcommand{\ninespace}{\vspace{9em}}
\newcommand{\tenspace}{\vspace{10em}}
\newcommand{\squeeze}{\vfill}


%-----------------------------------------------------------------------------------------------
%  BOLD LETTERS
%-----------------------------------------------------------------------------------------------
\newcommand{\bfa}{\textbf{a}}
\newcommand{\bfb}{\textbf{b}}
\newcommand{\bfc}{\textbf{c}}
\newcommand{\bfd}{\textbf{d}}
\newcommand{\bfe}{\textbf{e}}
\newcommand{\bff}{\textbf{f}}
\newcommand{\bfg}{\textbf{g}}
\newcommand{\bfh}{\textbf{h}}
\newcommand{\bfi}{\textbf{i}}
\newcommand{\bfj}{\textbf{j}}
\newcommand{\bfk}{\textbf{k}}
\newcommand{\bfl}{\textbf{l}}
\newcommand{\bfm}{\textbf{m}}
\newcommand{\bfn}{\textbf{n}}
\newcommand{\bfo}{\textbf{o}}
\newcommand{\bfp}{\textbf{p}}
\newcommand{\bfq}{\textbf{q}}
\newcommand{\bfr}{\textbf{r}}
\newcommand{\bfs}{\textbf{s}}
\newcommand{\bft}{\textbf{t}}
\newcommand{\bfu}{\textbf{u}}
\newcommand{\bfv}{\textbf{v}}
\newcommand{\bfx}{\textbf{x}}
\newcommand{\bfy}{\textbf{y}}
\newcommand{\bfz}{\textbf{z}}
\newcommand{\bfA}{\textbf{A}}
\newcommand{\bfB}{\textbf{B}}
\newcommand{\bfC}{\textbf{C}}
\newcommand{\bfD}{\textbf{D}}
\newcommand{\bfE}{\textbf{E}}
\newcommand{\bfF}{\textbf{F}}
\newcommand{\bfG}{\textbf{G}}
\newcommand{\bfH}{\textbf{H}}
\newcommand{\bfI}{\textbf{I}}
\newcommand{\bfJ}{\textbf{J}}
\newcommand{\bfK}{\textbf{K}}
\newcommand{\bfL}{\textbf{L}}
\newcommand{\bfM}{\textbf{M}}
\newcommand{\bfN}{\textbf{N}}
\newcommand{\bfO}{\textbf{O}}
\newcommand{\bfP}{\textbf{P}}
\newcommand{\bfQ}{\textbf{Q}}
\newcommand{\bfR}{\textbf{R}}
\newcommand{\bfS}{\textbf{S}}
\newcommand{\bfT}{\textbf{T}}
\newcommand{\bfU}{\textbf{U}}
\newcommand{\bfV}{\textbf{V}}
\newcommand{\bfW}{\textbf{W}}
\newcommand{\bfX}{\textbf{X}}
\newcommand{\bfY}{\textbf{Y}}
\newcommand{\bfZ}{\textbf{Z}}



\newcommand{\answers}[1]{{\vfill 
\begin{flushright}
{\it Answers:}\,#1
\end{flushright}
\medskip}} 

\newcommand{\flushitright}[1]{
\begin{flushright}
#1
\end{flushright}} 

%-----------------------------------------------------------------------------------------------
%  COUNTERS SLIDES
%-----------------------------------------------------------------------------------------------
\newcounter{applicationcounter}
\newcommand{\application}[1]{{\bf Application \addtocounter{applicationcounter}{1}\theapplicationcounter. }{#1}}
\newcounter{breakoutexamplecounter}
\newcommand{\breakout}[1]{{\bf Breakout \addtocounter{breakoutexamplecounter}{1}\thebreakoutexamplecounter. }{#1}}
\newcounter{conceptcounter}
\newcommand{\concept}[1]{{\bf Concept \addtocounter{conceptcounter}{1}\theconceptcounter. }{#1}}
\newcounter{foundationcounter}
\newcommand{\foundation}[1]{{\bf Foundation\addtocounter{foundationcounter}{1}\thefoundationcounter. }{#1}}
\newcounter{problemcounter}
\newcommand{\problem}[1]{{\bf Problem \addtocounter{problemcounter}{1}\theproblemcounter. }{#1}}
\newcounter{deconstructcounter}
\newcommand{\deconstruct}[1]{{\bf Deconstruct \addtocounter{deconstructcounter}{1}\thedeconstructcounter. }{#1}}

%-----------------------------------------------------------------------------------------------
%  COUNTERS OLDER
%-----------------------------------------------------------------------------------------------
\newcounter{examplecounter}
\newcommand{\myexample}[1]{{\bf Example \addtocounter{examplecounter}{1}\theexamplecounter. }{#1}}
\newcounter{interactivecounter}
\newcommand{\interactive}[1]{{\bf Interactive Ex. \addtocounter{interactivecounter}{1}\theinteractivecounter. }{#1}}
\newcounter{conceptualcounter}
\newcommand{\conceptual}[1]{{\bf Conceptual Qu. \addtocounter{conceptualcounter}{1}\theconceptualcounter. }{#1}}


